\documentclass{article}
\usepackage[utf8]{inputenc}
\usepackage[naustrian]{babel}
\usepackage{hyperref} 


\begin{document}

\title{Case study: Jenkins auf Azure Pipelines Migration}
\author{Simon Danglmaier}

\maketitle

\section*{Aufgabe}

\begin{itemize}
    \item Migration existierender Jenkins CI-Environment auf Azure Pipelines
    \item Laufender Betrieb muss gewährleistet sein
\end{itemize}

\section*{Plan}
\begin{enumerate}
    \item Vertrautmachen mit Jenkins \& Azure Konzepten
    \item Evaluierung \& Analyse von aktuellem Setup
    \item Falls möglich Aufgliederung in Subprojekte, die einzeln migriert werden können
          \begin{itemize}
              \item Core-Library
              \item Statische Analyse
          \end{itemize}
    \item Portierung eines minimalen Sub-Teils des Projekts
          \begin{itemize}
              \item Erlaubt Aufbau von Know-How
              \item Verifikation von portiertem Teil mit Resultat von alter Jenkins Pipeline und neuer Azure Pipeline
          \end{itemize}
    \item Portierung des restlichen Projekts
\end{enumerate}

\section*{Offene Fragen}

\begin{itemize}
    \item Erfahrungsberichte, Resource, Unterstützung aus anderen Departments: Haben andere Departments diese Umstellung bereits abgeschlossen?
    \item Anforderungen an neue Pipeline?
    \item Build-Agent Setup?
    \item Gibt es restriktive Zeitlimits, bis wann die Jenkins Pipeline migriert werden muss?
    \item Parallelbetrieb von alter und neuer Pipeline möglich?
    \item Dokumentation zu aktuellem Setup?
    \item Docker Builds?
\end{itemize}

\section*{Resourcen}

\begin{itemize}
    \item \href{https://learn.microsoft.com/en-us/azure/devops/pipelines/migrate/from-jenkins?view=azure-devops}{Microsoft Docs: Migrate from Jenkins to Azure Pipelines}
    \item \href{https://marketplace.visualstudio.com/items?itemName=ms-vsts.services-jenkins}{Azure DevOps Plugin: Jenkins Integration}
\end{itemize}

\end{document}